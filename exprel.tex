\documentclass[11pt,article,oneside]{memoir}
\input{vc}
\usepackage[wip]{apc-memoir}
\bibliography{apcmaster}

\chapterstyle{article-5}
\pagestyle{apcgit}

\title{Expressivism, Relativism, and the Problem of Communication}
\author{Alejandro P\'erez Carballo and Paolo Santorio}
\myaffiliation{University of Southern California and Australian National University}
\myemail{}
\version{1.0}
\gitstamp

\published{Partial working draft. Please do not circulate.}

\newcommand{\ought}{\largesquare}

\begin{document}
\maketitle
	
\noindent

\section{Introduction}

Suppose you're an antirealist about a certain domain of discourse---for example, normative discourse. You think that normative discourse is not on a par with descriptive discourse: normative claims don't report facts and they are not true or false, at least not in the same way as descriptive claims. How can you capture this in a general view of normative language?

Here is a first option: expressivism about normative discourse. According to expressivists, normative claims are not truth-apt: the notions of truth and falsity just don't apply to them. Moreover, normative claims don't express beliefs or representational attitudes in general, but rather a kind of conative mental state. Hence, Alejandro's utterance of ``Drowning babies in Coca Cola is okay" is not true or false and doesn't express a belief; rather, it expresses his endorsement of a normative standard, one that allows for the drowning of babies in Coca Cola. 

Here is a second option: truth relativism about normative discourse. This option involves granting that a notion of truth applies to normative claims. But they are not true or false absolutely; rather, they are only assigned a truth-value relative to an assessor. The very same claim, uttered in a unique determinate context, can be true for some assessors and false for others. Hence Paolo's utterance of  ``Drowning babies in Coca Cola is okay" at the current context---10:50 am on October 4th 2011, Canberra time---might be true for him and false for you.

How different are these two options? At the very least, they seem to diverge on one point: the relativist, but not the expressivist, appeals to a notion of truth. At the same time, expressivism and relativism about a certain domain of discourse share a number of features. In particular, both of them are hard-pressed to produce a non-standard account of communication. The intentions to transmit and acquire true beliefs seem to play a major role in any plausible account of our practice of assertion. Thus expressivists need to explain how we can have a practice of assertion involving claims that are non-truth-apt and attitudes that are non-representational. Similaly, relativists need to explain what is the point of transmitting and acquiring contents that are true at some circumstances and false of others---in particular, contents that might be true at the circumstances of the speaker, and false at the circumstances of the hearer. 

This paper has two goals. First, we focus on expressivism and show how expressivists should give an account of the problem of communication. Second, we argue that just the solution to this problem shows that expressivism and relativism are not distinct theories. More precisely, we argue for one constraint on any form of expressivism: the expressivist needs to have a notion of truth in her semantics; most plausibly, this notion is just the notion of truth used by the relativist. We leave it open whether all forms of truth relativism should be construed as brands of expressivism, though we suggest some considerations favoring this idea in the closing of our paper.

The claim that expressivism needs to appeal to truth might be taken as a major and perhaps fatal objection to it. This is not the spirit in which we make this claim. On the contrary, we regard this paper as a positive contribution towards building a general linguistic framework for antirealism. We don't think it's problematic that expressivism turns out not to be a distinct option. On the contrary, the convergence of different views into a unique framework is a sign of the viability and the strength of the latter. Hence, we hope that expressivists and relativists alike will welcome our conclusions. 

The paper proceeds as follows. In \autoref{epc}, we introduce expressivism and the problem of communication in more detail. In \autoref{coc}, we give our account of communication. In \autoref{dragon}, we ruthlessly kill a dragon.

\section{Expressivism and the problem of communication}\label{epc}

\subsection{Expressivism}
Throughout the paper, we take expressivism about normative discourse as our running example, though we intend our conclusions to apply, at least potentially, to expressivism about any kind of discourse.\footnote{PERHAPS HERE WE SHOULD HAVE SOME DISCLAIMERS/QUALIFICATIONS, ESPECIALLY ABOUT THE EPISTEMIC MODAL CASE.} Our starting point is a minimal conception of this type of expressivism, consisting of a negative and a positive claim: 
\begin{quote}
\textbf{Minimal Expressivism}
\begin{itemize}
\item[(a)] normative claims are not apt to describing, stating, or reporting facts; 
\item[(b)] normative claims express a non-cognitive (non-representational) attitude of some sort. 
\end{itemize}
\end{quote}
Of course, this minimal characterization needs to be fleshed out to be turned into a proper theory. One needs to say more about, first, the nature of the non-cognitive attitudes in play and their role in a general philosophy of mind; and second, about the expressivist's semantics for normative discourse. 

For present purposes, we assume the version of norm-expressivism developed by Allan Gibbard.\footcite{gibbard1990} The reasons are twofold. First, Gibbard's expressivism has a number of virtues. It is a very general theory of normative discourse; it yields a semantics that is compositional and fully compatible with standard syntactic views; it yields a simple account of logical consequence for normative discourse.\footnote{REFERENCES ON FREGE-GEACH PROBLEM.} Moreover, Gibbard's semantics for normative discourse dovetails well with the framework of communication we are going to use, namely the one developed by \citet{stalnaker1978}. 

Gibbard starts by introducing the attitude of accepting a norm: this is the kind of attitude that determines what individuals regard as mandated, permissible, or forbidden. Your judging that cannibalism is wrong amounts to your accepting a norm that forbids cannibalism; Alejandro's judging that drowning babies in Coca Cola is okay amounts to his accepting a norm that allows for the drowning of babies in Coca Cola. Gibbard doesn't define the notion of accepting a norm: rather, he assumes that this attitude will be part, on a par with beliefs and desires, of a (yet to come) fully developed empirical psychology. In particular, the notion of accepting a norm will play a crucial role in an evolutionary explanation of individuals' coordination in a social context.

How does this affect Gibbard's semantics for normative claims? Gibbard starts from a broadly Stalnakerian picture of linguistic content. On this picture, to make an assertion is, essentially, to distinguish among alternative possible ways the world might be.\marg{This needs to be revised. Worth keeping talk of content and talk of communication separate at this point, I think. (APC)} Making an assertion is claiming that the world is a certain way, rather than others. Hence, on Stalnaker's picture, the contents of assertions are just sets of possible worlds, i.e.~the worlds that are not ruled out by the assertion. Of course, Gibbard cannot help himself to this picture as it is: for the expressivist, normative claims don't distinguish among possible ways the world might be. Gibbard's move is to introduce a new kind of possibilities in the Stalnaker framework. Descriptive claims distinguish among possible worlds; normative claims distinguish among fully specified normative standards, or, as Gibbard calls them, \emph{systems of norms}. Hence the contents of 

\ex. Paolo drowns babies in Coca Cola.

\ex. Drowning babies in Coca Cola is okay.

are given by, respectively:

\ex.[] $\lbrace$$w$: Paolo drowns babies in Coca Cola in $w$$\rbrace$

\ex.[] $\lbrace$$n$: $n$ allows drowning babies in Coca Cola$\rbrace$

For several reasons, in the formalism it's useful to treat all clauses as having contents of the same kind. Gibbard can easily do this: he takes all contents to be, formally, sets of world-norm pairs. Descriptive claims impose constraints only on the world component. Normative claims impose constraints only on the norm component. Hence, from a formal standpoint, the contents of the two claims above turn out to be:

\ex.[] $\lbrace$$\langle w, n \rangle$: Paolo drowns babies in Coca Cola in $w$$\rbrace$

\ex.[] $\lbrace$$\langle w, n \rangle$: $n$ allows drowning babies in Coca Cola$\rbrace$


\subsection{Communication}

We will think of conversations as taking place against a background body of information. For our purposes, we can think of this body of information---the \emph{context set}---as representing what is mutually recognized as being taken for granted for the purposes of the conversation. The purpose of an assertion, on this picture, is to add information to the context set.\footnote{This picture is due, in its essentials, to Stalnaker. See  e.g. \citealt{stalnaker1973,stalnaker1978}. See also \citealt{lewis1979a}.}

For concreteness, we make a few working assumptions. First, we will think of a body of information as represented by a set of \emph{possible worlds}: the worlds that are not ruled out by that body of information. Second, we think of the \emph{content} of a declarative sentence as (determining) a set of possible worlds: those worlds in which the sentence is true. 

As the conversation evolves, so does the context set. After a successful utterance of a sentence $s$, the content of $s$ gets added to the context set---the context set is shrunk to exclude those in which $s$ is false.\footnote{And some more: the context set will, in general, get updated so that every world in it is one in which an utterance took place, and so on. This need not be the case: if participants are engaged in pretense, describing a situation in which their conversation is not taking place, information about the conversation will not get added to the context set. For our purposes, we can engage in the pretense that such conversations never take place.} 

Consider a simple example. Sue and Tom are having a conversation. The main item on the agenda: what Paolo has been up to in the past year. So far, they take each other to presuppose that Paolo lived in Cambridge last year. We can thus model the body of information they take for granted for the purposes of the conversation as the set of all worlds in which Paolo lived in Cambridge last year. Now, suppose Tom utters the sentence `Paolo did not pay taxes in Cambridge last year.' Assuming Tom's assertion is successful, the context set will now get updated with that piece of information. The updated context set will now contain all those worlds in which Paolo lived in Cambridge last year without paying taxes. 


\subsection{Putting things together}

Expressivists have seemed partial to the above picture of communication: not to the letter, mind you, but to the spirit.\footnote{[\textsc{insert references here}]} Their idea is quite simple. In a conversation about normative matters we are trying to influence what systems of norms others accept. Thus, when engaged in a conversation, there are now two parameters that we need to keep track of: the worlds compatible with what is mutually taken for granted for the purposes of a conversation, and the systems of norms compatible with what is mutually recognized as being accepted by participants in the conversation.\footnote{In the terminology \citealt{lewis1979}, expressivists think of the \emph{conversational score} of a conversation as determining (at least) a pair consisting of a set of worlds and a set of norms.} 

Go back to Sue and Tom. Their conversation has now moved on to issues of greater significance. They have moved on to discuss the moral status of a few types of actions. So far, they take each other to accept that drowning babies in Coca-Cola is wrong. We can thus model what is mutually recognized as accepted for the purposes of the conversation by a set of systems norms: those systems of norms that forbid drowning babies in Coca-Cola. The moral status of tax-evasion is not one that has been discussed yet, but Sue goes on to utter the sentence: `Tax-evasion is wrong'. If Tom does not object, the updated context set will now include only systems of norms that forbid tax-evasion. Every system of norms in that set will now forbid drowning babies in Coca-Cola and not paying taxes. 

When giving a full-theory of communication, expressivists will probably add some subtleties to this picture. They might want to model the context set as a set of \emph{pairs}---those world-norm pairs $\langle w, n \rangle$ such that Paolo lived in Cambridge in $w$ without paying taxes \emph{and} $n$ forbids drowning babies in Coca-Cola and not paying taxes. They can thus predict, as they should, that Tom and Sue now take for granted, for the purposes of the conversation, that something Paolo did was wrong.\footnote{[\textsc{insert reference to the wishful thinking objection here}].}

For our purposes, however, we can set those subtleties aside. For the questions we will raise for expressivism would arise even if we never engaged in conversation to exchange information about what the world is like.  

\subsection{The problem of communication}

An important task for any theory of communication is to make sense of the practice of conversation: to explain why communication takes place the way it does. For the case of descriptive discourse, the above account of communication can give such an explanation. In broad outline, the explanation could go like this:

\begin{quote}
There is a convention in place so that (\emph{i}) when I utter a sentence $s$, I thereby convey that I believe the content of $s$---what $s$ says; and (\emph{ii}) if you take me to be reliable on the subject matter at hand, you will take my believing that $s$ is true as a reason for you to believe that $s$ is true.\footnote{Cf. the `convention of truthfulness and trust' in \citealt{lewis1975}.} Since this is common knowledge, a successful assertion of $s$ will result in the ruling out of those worlds in which $s$ is false---it will now be mutually recognized that we are taking for granted, at least for the purposes of the conversation, that $s$ is true.
\end{quote}

Can the expressivist give a similar story for the case of \emph{normative} discourse? Can she make sense of the practice of assertion of normative claims? 

The expressivist cannot take the above explanation verbatim. After all, it relies on the notion of \emph{reliability}, and it is not obvious how an expressivist can make sense of this notion. Reliability is typically cashed out in terms of correspondence with how things are. You take me to be reliable if you take my believing that so and so as an indication that so and so is the case. You take Sue's beliefs about some subject matter to be a good guide to the truth, because you take the following schema to be generally true (with $p$ ranging over a suitably restricted domain): 

\ex.[] If Sue believes $p$, then $p$ is true.

It may be that expressivists can ultimately appeal to something much like reliability in making sense of our communicative practices. But pending such a story, the expressivist needs a different way of making sense of our practice of assertion. 

Although no alternative account has been given in full detail, expressivists have gestured toward a way of making sense of communication, one that generalizes to descriptive and normative discourse alike. The idea, introduced by Allan Gibbard, is to think of communication as primarily an exercise in \emph{coordination}. We are social creatures: we engage in conversation because coordinating on some particular family of attitudes---beliefs, say, or normative acceptances---is likely to help us meet our goals. 

To a first approximation, this seems like a good strategy for making sense of our communicative practices. The point of engaging in conversation about non-normative matters is to coordinate on a body of beliefs. The point of engaging in conversation about normative matters is to coordinate on a system of norms. We work out, as a community, what to think about the world. We work out, as a community, what system of norms to accept. Communication can be understood as a way to foster such coordination. 

Here is Allan Gibbard, explicitly making this point: 

\begin{small}
\begin{quote}
	The biological function of the mechanisms underlying our normative capacities is to coordinate. Hence the psychic mechanisms that produce normative judgments are not systems of natural representations, they are coordinating systems. Their biological function is not to put something in the head in correspondence with their subject matter; it is to coordinate what is in one person's head with what is in another's. \cite[p. 110]{gibbard1990}
\end{quote}
\end{small}
And Seth Yalcin, building on Gibbard's strategy to flesh out an account of communication friendly to his expressivist account of epistemic modal talk: 
\begin{small}
	\begin{quote}
[I]n modeling the communicative impact of an epistemic possibility claim, we construe the objective as one of coordination on a certain global property of one’s state of mind—the property of being compatible with a certain proposition—not one of coordination concerning the way the world is. \cite[p. 310]{yalcin2011}
	\end{quote}
\end{small}

Unfortunately, to account for \emph{our} communicative practices, coordination by itself will not do. The reason is that coordination is too cheap. If our goal in conversation is to coordinate on one of two systems of norms, flipping a coin should do just fine.  Mere coordination is what we are after when deciding what side on the road to drive on: there, a coin flip would have done the trick. There, what we want is that we all drive on the same side of the road, regardless of \emph{which} side that is. But we don't normally rely on coin flips when working out what to do, no more so than we rely on coin flips when working out what the world is like. 

This is not to say that coordination cannot play a part in a general account of communication, one that makes sense of both normative and non-normative discourse. But a full account must do more than point out the advantages of coordination, and the ways in which conversation can foster coordination. For there are much more efficient ways of achieving coordination, so a desire for coordination on our beliefs or our normative acceptances cannot explain the complexities of our communicative practices. In the next section, we look at the ways in which this idea of communication as coordination needs to be constrained if it is to yield an explanation of why communication works the way it does. 

% TODO 
%(fold) 
% Gibbard thinks coordination won't do: "This accounts for pressures towards consensus in ethical matters. For many sorts of normative judgments, though, such an account of disagreement isn’t to be had. Two egoists can agree that each ought rationally to act to frustrate the aims of the other, but this agreement is far from fostering coordination. A story rooted in coordination won’t work for belief either, since we don’t standardly need to coordinate our beliefs." Found here: http://philosophy.fas.nyu.edu/docs/IO/5882/GibbardPaper.pdf
%(end)



% On this picture, communication can be thought of as a matter of coordination on a body of information.

\section{Constraints on coordination}\label{coc}

In this section, we argue that expressivists are committed to the following claim: 

\ex.[(\textsc{c})] In any `normal' conversation, it is presupposed that there is a (unique) system of norms on which the participants' acceptances ought to converge.

A few remarks are in order. First, \emph{presupposition} here does not require unconditional acceptance. All the expressivist is committed to is that, while the conversation takes place, participants must take for granted \emph{for the purposes of the conversation} that there is a unique system of norms to which the conversation ought to converge. 

Second, (\textsc{c}) does not amount to a capitulation to a form of realism or objectivism. A realist will presumably endorse the following thesis: 

\ex.[(\textsc{c$^*$})] There is a unique system of norms (the `true' system of norms) to which, in any `normal' conversation, participants' acceptances ought to converge. 

But (\textsc{c$^*$}) is crucially distinct from (\textsc{c}), and neither one entails the other. It may be that there is a stronger version of (\textsc{c$^*$}) that realists are also committed to, e.g.

\ex.[] There is a unique system of norms such that, in any `normal' conversation, it is presupposed that the participants' acceptances ought to converge to that system of norms. 

And while this principle does entail (\textsc{c}), it is not entailed by it. 

Third, note that in endorsing (\textsc{c}) the expressivist need not impute ordinary speakers with any type of false belief. In presupposing that there is a unique system of norms to which our acceptances \emph{ought} to accept I am not, according to the expressivist, taking a stance on what the world is like. So the presupposition in (\textsc{c}) is compatible with a perfectly accurate view on what the world is like. 

Finally, a word on the qualifier `normal': by `normal' conversation (at \emph{t}) we mean one in which it is presupposed (at \emph{t}) that there is a point to communication. In other words, one in which it is presupposed that participants in the conversation intend to achieve coordination.

Our argument for (\textsc{c}) is somewhat complex. Before spelling it out in some detail, we will take a moment to briefly outline the main moves we will make along the way. 

\subsection{Overview of our main argument}

We use `$n$',`$s$', and `$c$' as variables ranging over systems of norms, participants in a conversation, and context sets, respectively. Also, we use `$\mathsf{Acc}_s$' to denote the set of systems of norms compatible with what $s$ accepts, and we use `$\ought$' to abbreviate the \emph{ought} operator. This allows us to introduce some abbreviations. Our conclusion, for example, is just the claim that the following is a presupposition of every `normal' conversation

\ex.[] \textsc{convergence}: $(\exists n)(\forall s) \ought (n \in \mathsf{Acc}_s)$


In arguing that \Last must be presupposed in any conversation,\footnote{Henceforth, we drop the qualifier `normal' for the sake of readability.} we rely on two main premises. The first is that the following is presupposed in every conversation with context set $c$:

\ex.[(\textsc{p1})] $\ought ((\exists c' \subset c)(\forall s)\mathsf{Acc}_s \subset c')$.

The second is that 

\ex.[(\textsc{p2})] $(\exists n)(\forall s)\ought(n \in \mathsf{Acc}_s)$

is also presupposed in any conversation. 

In plain English, the first premise is that, in every conversation, it is presupposed that the context set ought to be narrowed down. The second premise is that, in every conversation, it is presupposed that there is a system of norms that ought not be ruled out from any of the participant's acceptance state. 

Unsurprisingly, most of the action will take place in the argument that (\textsc{p2}) is a presupposition in any normal conversation. 

Once we establish that (\textsc{p1}) and (\textsc{p2}) must be presupposed in any conversation, we will have completed the hard work. For, as we will show, every context set that contains \textsc{p1} and \textsc{p2} is a context set that contains \textsc{convergence}. Thus, if (\textsc{p1}) and (\textsc{p2}) are presupposed in every conversation, our conclusion (\textsc{c}) must obtain. 

\subsection{The first premise}

Our first premise is the least controversial claim we make here. In essence, it amounts to no more than the claim that when engaged in conversation, we take for granted that we ought to agree on one among a number of live alternatives. The point of conversation, as it were, is to achieve some consensus as to which of the live alternatives is the one we ought to coordinate on. When engaged in conversation, we presuppose that we ought to rule out some of the relevant alternatives. 

It is important to be clear on what the scope of our claim is. We are not claiming that an agent ought to think that `uncertainty' ought to be eliminated. In other words, we are not claiming that whenever an agent is not settled as to what to do, she must think she ought to come to a conclusion. 

Rather, our claim is that \emph{while} we take there to be a point in engaging in conversation, we must think that we ought to rule out some alternatives from the context set. For the point of conversation, we are supposing, is to rule out alternative answers to questions about what to do, in order to coordinate on a unique answer to such questions. 

This is not to deny that a conversation can reach a stalemate. But once we no longer think we ought to rule out some alternatives---either because we've coordinated on a particular system of norms, or because we realize that our attempt at coordinating is unlikely to succeed---we will think there is no longer any point to engaging in conversation. 

\subsection{The second premise}

Our second premise, then, is what does the main theoretical work in the argument. This premise states that, in every conversation, there is a norm that that ought not be ruled out from the acceptance states of the participants. At an intuitive level, this premise can be regarded as a minimal requirement of objectivity. Speakers must presuppose that at least one of the possibilities that they regard as open ought not be ruled out. Notice that this is much weaker than the claim that speakers' attitudes should converge on a unique norm. It only requires that they keep regarding one of the possibilities as open: i.e., that they not be allowed to accept normative propositions incompatible with it. 

Even though our second premise seems rather weak, it requires extensive argument. In particular, we are going to establish it via two intermediate claims, each of which requires independent support. Here is the overall strategy.

For simplicity, we're going to consider the special case of a conversation between two speakers $A$ and $B$. The argument can be easily generalized to conversations involving a greater number of speakers. The first claim we want to establish is that, in any conversation, speaker $A$ accepts that speaker $B$ accepts that speaker $A$ accepts that there is one norm that ought not be ruled out. In symbols:

\ex.[(A1)] $\Ac_A \Ac_B (\Ac_A (\exists n \ought (n \in {Acc}_A)))$.

From here, we go on to argue for the claim that each speaker in the conversation accepts that there is a norm she ought not rule out. In symbols:

\ex.[(A2)] $\Ac_A (\exists n \ought (n \in {Acc}_A))$.

From (A2), and from the corresponding principle concerning $B$, we're going to argue for the claim that speakers accept that there is a unique norm that they \emph{both} ought not rule out. This will establish our second premise, (\textsc{p2}). 


\subsection{The argument for \textsc{p2}}
\textbf{The argument for (A1).} Our first aim is then to establish:

\ex.[(A1)] $\Ac_A \Ac_B (\Ac_A (\exists n \ought (n \in {Acc}_A)))$.

To do this, we proceed in steps. First we're going to show that $B$ accepts there is a norm $A$ ought not rule out. In symbols:

\ex.[($\alpha$)] $\Ac_B (\exists n \ought (n \in {Acc}_A))$.

(where, as usual, `$\ought$' is shorthand for a deontic ought). Here's the argument: suppose that there is a norm $n^*$ in B's acceptance state according to which $A$ may rule out any norm whatsoever. I.e., $B$ leaves open the possibility that it's okay for $A$ to rule out any norm, and hence come to accept any normative proposition whatsoever. But now, this would be an `anything goes' situation where $A$ is allowed to rule out anything that $B$ accepts, even if $A$ thinks that there's nothing wrong with $B$'s asserting or accepting it. In this kind of situation, the following conversation could take place:
\begin{itemize}
\item[B:] Cannibalism is wrong.
\item[A:] Nothing wrong with that, yet I still accept that cannibalism is okay. [Goes back to munching on human flesh.]
\end{itemize}
Notice that $A$ is not rejecting what $B$ says. More in general, she need not find anything wrong with $B$'s assertions of normative claims. But she is not constrained in any way by $B$'s utterances, since she's allowed to accept any normative proposition. Hence she can just stick to her acceptances; $B$'s claims will have no effect on her. 

We claim that, in this kind of situation, it would make no sense for $B$ to engage in communication; there would be no point to it. Hence, if $B$ agrees to engage in communication with $A$, she must do so under the assumption that it's not okay for $A$ to accept any normative proposition whatsoever. 

Let us address two worries. First, you may worry that our argument involves a kind of scope equivocation. We've argued for the claim that $B$ must accept that it's not the case that anything goes for $A$:
\begin{center}
$\Ac_B \lnot (\forall n (\sf{May} (n \not \in {Acc}_A))$
\end{center}
But, the objection goes, the argument only shows that $B$ must not accept that anything goes for $A$:
\begin{center}
$\lnot \Ac_B (\forall n (\sf{May} (n \not \in {Acc}_A))$ 
\end{center}
The reasoning is this: if $B$ is convinced that for $A$ anything goes, then it's true that it makes no sense for her to engage in communication. But, as long as $B$ leaves open the possibility that $A$ is subject to some constraints in what she accepts, it makes sense for her to at least try to engage in conversation with her.

The objection confuses acceptance with belief. We grant that $B$ may not \emph{believe} that it's not true that anything goes for $A$. Nevertheless, for the purposes of conversation, she must take that for granted. To make an assertion is not to \emph{try} to make a demand that the hearer changes her attitude. Rather, it just is to make a demand to that effect. By making an assertion, the speaker represents herself as accepting that it's not true that anything goes. Hence she takes that for granted for the purposes of the conversation. 

The second worry puts pressure on the very core of our argument. We claim that, in a situation where anything goes for $A$, it makes no sense for $B$ to engage in communication with $A$. But this might seem too strong. Consider other instances of coordination, like---again---choosing to drive on one side of the road rather than another. This case arguably does display the case of permissibility that we want to deny for the case of communication: there are no good reasons to pick one side rather than another. Yet it makes sense for agents to make proposals about coordination. For example, B could propose to A: ``Let's drive on the left-hand side of the road". Where is the disanalogy?

[This is the infamous Gabe Rabin worry. I have ideas about the reply, but I'd like to discuss them with you beforehand.]

Now, let us move on with the argument. Once we've established ($\alpha$), we can run through similar reasoning to establish a different claim, namely: B accepts that A herself accepts that there is a norm that A ought not rule out.\footnote{There are modes-of-presentation-kind issues with this formulation. What we want to say is that B accepts that A accepts that there is a norm that \emph{she herself}, individuated in the self-locating way (see \citet{perry1979} and \citet{lewis1979} for the notion of self-locating belief), ought not rule out. These issues are orthogonal to our main concern here and can be put on the side.} In symbols, this amounts to: 

\ex.[($\beta$)] $\Ac_B (\Ac_A (\exists n \ought (n \in \mathsf{Acc}_A)))$

Notice the difference between ($\alpha$) and ($\beta$). ($\alpha$) claims that B presupposes a normative proposition---something about what ought to be the case. ($\beta$), on the contrary, says that B presupposes a descriptive proposition: in particular, she presupposes something about A's attitudes. Thus the argument for ($\beta$) must appeal to worlds instead of norms. 

Aside from this difference, the arguments are basically parallel.\footnote{Indeed, the whole rationale behind introducing the argument for ($\alpha$) was giving a simpler and more intuitive version of the argument for ($\beta$). Notice that we don't use ($\alpha$) while establishing ($\beta$). \marg{DIALECTICALLY, THIS IS A BIT FUNNY AND IT DID ARISE SOME PUZZLEMENT DURING THE TALK. WE SHOULD TALK ABOUT WHAT TO DO ABOUT IT. EVENTUALLY, I THINK IT MIGHT BE MORE ECONOMICAL TO JUST HAVE ($\beta$) AND GET RID OF THE REST.]}} Suppose there is a world $w^*$ compatible with B's acceptances in which A doesn't accept that there is a norm A ought not rule out. In $w^*$, A will not accept that there is a norm she ought to accept. So A might think that it is permissible to accept any normative proposition. In that world, by the reasons explained above, it makes no sense for B to engage in communication. Hence, if B has to accept that it makes sense for her to engage in conversation with A, she must take for granted, for the purposes of the conversation, that A accepts that it's not the case that anything goes for her. This establishes ($\beta$). 

Once we have ($\beta$), it's easy to get to our first aim, (A1). If A is rational,\marg{[WE SHOULD PROBABLY SPELL OUT SOMEWHERE THE KIND OF ASSUMPTION ABOUT RATIONALITY WE NEED]} she will be able to go through the argument explained above and see that ($\beta$) is required for communication to function. Hence she will accept ($\beta$), which gives us just (A1):

\ex.[(A1)] $\Ac_A \Ac_B \Ac_A (\exists n \ought (n \in \mathsf{Acc}_A))$.

\vspace{1\baselineskip}

\textbf{The argument for (A2).} The next step consists in arguing for the claim that each speaker accepts that there is a norm that she ought not rule out. In symbols, this involves eliminating two iterations of the `$Acc$' operator from (A1):

\ex.[(A2)] $\Ac_A (\exists n \ought (n \in {Acc}_A))$.

The argument is pretty linear. Suppose that, for any proposition (descriptive or normative) $\phi$, we have both that B accepts that A accepts $\phi$, and that A accepts that B accepts that A accepts that $\phi$:

\ex.[(i)] $\Ac_B \Ac_A \phi$ 


\ex.[(ii)] $\Ac_A \Ac_B \Ac_A \phi$

Now, assume that, despite this, A doesn't accept $\phi$, i.e.~$\neg \Ac_A \phi$. Then A has a choice. 
\begin{itemize}
\item She can either point out that actually she does not accept A. 
\item Or she can act as if she accepts it for the purposes of the conversation.
\end{itemize}
If she goes for the first option, in the next conversational turn B will stop accepting that A accepts $\phi$. If she goes for the second option, she will actually come to accept A for the purposes of the conversation. (Notice that here, once more, the particular nature of acceptance and its divergence from belief is crucial to make the point.)

Now, our claims ($\beta$) and (A2) are just instances of schemas (i) and (ii). We established above that ($\beta$) and (A2) are going to hold with respect to any `normal' conversation. Hence, in any normal conversation, speakers are faced, at any point, with the alternative of denying that they accept that there is a norm they ought not to rule out, or going along with it. If they make the former choice, then they will essentially opt out of the conversation. In that case, in fact, ($\beta$) will stop holding and communication will lose its point. Hence, if the conversation goes on, then they effectively accept that there is a norm they ought not to rule out.\marg{[IN THE HANDOUT WE HAD THE CLAIM THAT (A2) AND (A3) CAN FAIL TO HOLD IN AT MOST ONE CONVERSATIONAL TURN. I'M NOT REALLY SURE THAT THE DIACHRONIC IDEA IS THE RIGHT WAY TO PUT IT.]}

In symbols, by running the reasoning for both speakers, we get the following two claims:

\ex.[(A2)] $\Ac_A (\exists n \ought (n \in \mathsf{Acc}_A))$.

\ex.[(A3)] $\Ac_B (\exists n \ought (n \in \mathsf{Acc}_B))$.

\vspace{1\baselineskip}

\textbf{The argument from (A2) and (A3) to (\textsc{p2}).} Let us take stock. We've argued that each participant in a conversation accepts that there is a norm that she ought not rule out. To get (\textsc{p2}), we need to show that they all accept that there is a \textit{unique} norm that they \textit{both} ought not rule out. 

The argument for this conclusion is not dissimilar from the argument we gave for ($\alpha$). Suppose that it's not true that speakers accept that there is a norm they both ought to accept. For example, suppose that B doesn't accept that. By the previous argument, we know that B accepts that there is a norm that she ought not rule out: call that norm `$n^*$'. According to B, it might be that A is allowed to rule out $n^*$. This allows for a situation of the following kind. Suppose that A utters a normative claim: for example, `Drowning babies in Coca Cola is okay'. Now, B may accept that A's assertion is correct in a number of respects: it is sincere, it doesn't violate any constraints on A's attitudes, etc. Nevertheless, since B doesn't accept that she and her interlocutor are coordinating on the same norm, she has no reason to assent to A's assertion. It might be that the proposition expressed by `Drowning babies in Coca Cola' is okay to accept for A, but not for B. I.e., it might rule out $n^*$, the norm that B ought not rule out. 

If speakers don't accept that there is a unique norm that they ought not rule out, they cannot trust each other. It might be that one utters a sentence that is perfectly okay by her standards, but is unacceptable for the hearer. But trust is essential for successful communication (see, among others, \citet{lewis[convention]} and \citet{lewis[lang&lang]}). Hence, if speakers want to engage in communication with each other, they will accept that there is a unique norm they ought to converge on. Hence, if (A2) and (A3) hold, (\textsc{p2}) is going to be in the context set of any normal conversation. In symbols, this amounts to claiming that the following holds in any normal context set:

\ex.[(\textsc{p2})] $(\exists n)(\forall s)\ought(n \in \mathsf{Acc}_s)$

\vspace{1\baselineskip}

\textbf{Summary.} It's time to take stock. We have bootstrapped our way into our conclusion from a number of intermediate premises. In essence, we have claimed, first, that while engaging in conversation each speaker must assume that `not anything goes' for her interlocutor. From here, we have argued, exploiting the peculiar public nature of acceptance, that each speaker must assume that `not anything goes' for her own attitudes as well. Finally, we have argued that, if speakers have to trust each other, they must assume that `not anything goes' for them in the same way---essentially, that the normative constraints applying to their attitudes are the same.

The main assumptions in the foregoing have been the notion of acceptance, and in particular its public nature, and the claim that speakers need to trust each other to engage in communication. Both seem to us very minimal requirement. If this is right, and if the argument is valid, then a minimal objectivity requirement---i.e., (\textsc{p2})---is in place with regard to all normal conversations. 

\subsection{Putting \textsc{p1} and \textsc{p2} together}
We have argued that these two normative propositions are in the context set of any normal conversation:

\ex.[(\textsc{p1})] $\ought ((\exists c' \subset c)(\forall s)\mathsf{Acc}_s \subset c')$.

\ex.[(\textsc{p2})] $(\exists n)(\forall s)\ought(n \in \mathsf{Acc}_s)$

The first amount to the claim that the context set should be shrunk: some live possibilities should be eliminated. The second amounts to the claim that there is at least one element in the context set that ought not be ruled out. It is not hard to prove that (\textsc{p1}) and (\textsc{p2}) together entail ($c^*$):\footnote{See the appendix for a formal proof.\marg{I'm assuming that we're going to have a proof of this in an appendix, just for completeness, but we should discuss this.}
}

\a.[($c^*$)] $(\exists n) (  \forall s)  \ought( \{ n\} =  \mathsf{Acc_s})$. 

The intuitive reasoning is quite simple: if, at any point in the context set, there is a constraint to shrink it,\marg{Probably something slightly stronger is needed, namely: at ANY POINT in the future of the context set we should keep shrinking. We should talk about this too.} 
and if there is a constraint that one of the norms ought not be ruled out, then effectively the context set ought to converge on a unique norm. 

Once again, we've argued that ($c^*$) ought to be in any normal context set. We have not argued that it is true. Indeed, it's not clear how the question of truth and falsity of normative claims like ($c^*$) should be handled by an expressivist theorist; but me need make no commitments about this. Our point is that speakers' endorsement of ($c^*$) is a precondition on their engaging in communication about normative claims. As we're going to point out, this is enough to require a notion of truth in the expressivist's semantics. 


\section{Morals}\label{dragon}

 

\printbibliography
\end{document}