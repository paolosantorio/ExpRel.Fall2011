\documentclass[11pt,article,oneside]{memoir}
\input{vc}
\usepackage[wip]{apc-memoir}
\bibliography{apcmaster}

\chapterstyle{article-5}
\pagestyle{apcgit}

\title{Expressivism, Relativism, and the Problem of Communication}
\author{Alejandro P\'erez Carballo and Paolo Santorio}
\myaffiliation{University of Southern California and Australian National University}
\myemail{}
\version{1.0}
\gitstamp

\published{Partial working draft. Please do not circulate.}

\newcommand{\ought}{\largesquare}

\begin{document}
\maketitle
	
\noindent

\section{Introduction}

Suppose you're an antirealist about a certain domain of discourse---for example, normative discourse. You think that normative discourse is not on a par with descriptive discourse: normative claims don't report facts and they are not true or false, at least not in the same way as descriptive claims. How can you capture this in a general view of normative language?

Here is a first option: expressivism about normative discourse. According to expressivists, normative claims are not truth-apt: the notions of truth and falsity just don't apply to them. Moreover, normative claims don't express beliefs or representational attitudes in general, but rather a kind of conative mental state. Hence, Alejandro's utterance of ``Drowning babies in Coca Cola is okay" is not true or false and doesn't express a belief; rather, it expresses his endorsement of a normative standard, one that allows for the drowning of babies in Coca Cola. 

Here is a second option: truth relativism about normative discourse. This option involves granting that a notion of truth applies to normative claims. But they are not true or false absolutely; rather, they are only assigned a truth-value relative to an assessor. The very same claim, uttered in a unique determinate context, can be true for some assessors and false for others. Hence Paolo's utterance of  ``Drowning babies in Coca Cola is okay" at the current context---10:50 am on October 4th 2011, Canberra time---might be true for him and false for you.

How different are these two options? At the very least, they seem to diverge on one point: the relativist, but not the expressivist, appeals to a notion of truth. At the same time, expressivism and relativism about a certain domain of discourse share a number of features. In particular, both of them are hard-pressed to produce a non-standard account of communication. The intentions to transmit and acquire true beliefs seem to play a major role in any plausible account of our practice of assertion. Thus expressivists need to explain how we can have a practice of assertion involving claims that are non-truth-apt and attitudes that are non-representational. Similaly, relativists need to explain what is the point of transmitting and acquiring contents that are true at some circumstances and false of others---in particular, contents that might be true at the circumstances of the speaker, and false at the circumstances of the hearer. 

This paper has two goals. First, we focus on expressivism and show how expressivists should give an account of the problem of communication. Second, we argue that just the solution to this problem shows that expressivism and relativism are not distinct theories. More precisely, we argue for one constraint on any form of expressivism: the expressivist needs to have a notion of truth in her semantics; most plausibly, this notion is just the notion of truth used by the relativist. We leave it open whether all forms of truth relativism should be construed as brands of expressivism, though we suggest some considerations favoring this idea in the closing of our paper.

The claim that expressivism needs to appeal to truth might be taken as a major and perhaps fatal objection to it. This is not the spirit in which we make this claim. On the contrary, we regard this paper as a positive contribution towards building a general linguistic framework for antirealism. We don't think it's problematic that expressivism turns out not to be a distinct option. On the contrary, the convergence of different views into a unique framework is a sign of the viability and the strength of the latter. Hence, we hope that expressivists and relativists alike will welcome our conclusions. 

The paper proceeds as follows. In \autoref{epc}, we introduce expressivism and the problem of communication in more detail. In \autoref{coc}, we give our account of communication. In \autoref{dragon}, we ruthlessly kill a dragon.

\section{Expressivism and the problem of communication}\label{epc}

\subsection{Expressivism}
Throughout the paper, we take expressivism about normative discourse as our running example, though we intend our conclusions to apply, at least potentially, to expressivism about any kind of discourse.\footnote{PERHAPS HERE WE SHOULD HAVE SOME DISCLAIMERS/QUALIFICATIONS, ESPECIALLY ABOUT THE EPISTEMIC MODAL CASE.} Our starting point is a minimal conception of this type of expressivism, consisting of a negative and a positive claim: 
\begin{quote}
\textbf{Minimal Expressivism}
\begin{itemize}
\item[(a)] normative claims are not apt to describing, stating, or reporting facts; 
\item[(b)] normative claims express a non-cognitive (non-representational) attitude of some sort. 
\end{itemize}
\end{quote}
Of course, this minimal characterization needs to be fleshed out to be turned into a proper theory. One needs to say more about, first, the nature of the non-cognitive attitudes in play and their role in a general philosophy of mind; and second, about the expressivist's semantics for normative discourse. 

For present purposes, we assume the version of norm-expressivism developed by Allan Gibbard.\footcite{gibbard1990} The reasons are twofold. First, Gibbard's expressivism has a number of virtues. It is a very general theory of normative discourse; it yields a semantics that is compositional and fully compatible with standard syntactic views; it yields a simple account of logical consequence for normative discourse.\footnote{REFERENCES ON FREGE-GEACH PROBLEM.} Moreover, Gibbard's semantics for normative discourse dovetails well with the framework of communication we are going to use, namely the one developed by \citet{stalnaker1978}. 

Gibbard starts by introducing the attitude of accepting a norm: this is the kind of attitude that determines what individuals regard as mandated, permissible, or forbidden. Your judging that cannibalism is wrong amounts to your accepting a norm that forbids cannibalism; Alejandro's judging that drowning babies in Coca Cola is okay amounts to his accepting a norm that allows for the drowning of babies in Coca Cola. Gibbard doesn't define the notion of accepting a norm: rather, he assumes that this attitude will be part, on a par with beliefs and desires, of a (yet to come) fully developed empirical psychology. In particular, the notion of accepting a norm will play a crucial role in an evolutionary explanation of individuals' coordination in a social context.

How does this affect Gibbard's semantics for normative claims? Gibbard starts from a broadly Stalnakerian picture of linguistic content. On this picture, to make an assertion is, essentially, to distinguish among alternative possible ways the world might be.\marg{This needs to be revised. Worth keeping talk of content and talk of communication separate at this point, I think. (APC)} Making an assertion is claiming that the world is a certain way, rather than others. Hence, on Stalnaker's picture, the contents of assertions are just sets of possible worlds, i.e.~the worlds that are not ruled out by the assertion. Of course, Gibbard cannot help himself to this picture as it is: for the expressivist, normative claims don't distinguish among possible ways the world might be. Gibbard's move is to introduce a new kind of possibilities in the Stalnaker framework. Descriptive claims distinguish among possible worlds; normative claims distinguish among fully specified normative standards, or, as Gibbard calls them, \emph{systems of norms}. Hence the contents of 

\ex. Paolo drowns babies in Coca Cola.

\ex. Drowning babies in Coca Cola is okay.

are given by, respectively:

\ex.[] $\lbrace$$w$: Paolo drowns babies in Coca Cola in $w$$\rbrace$

\ex.[] $\lbrace$$n$: $n$ allows drowning babies in Coca Cola$\rbrace$

For several reasons, in the formalism it's useful to treat all clauses as having contents of the same kind. Gibbard can easily do this: he takes all contents to be, formally, sets of world-norm pairs. Descriptive claims impose constraints only on the world component. Normative claims impose constraints only on the norm component. Hence, from a formal standpoint, the contents of the two claims above turn out to be:

\ex.[] $\lbrace$$\langle w, n \rangle$: Paolo drowns babies in Coca Cola in $w$$\rbrace$

\ex.[] $\lbrace$$\langle w, n \rangle$: $n$ allows drowning babies in Coca Cola$\rbrace$


\subsection{Communication}

We will think of conversations as taking place against a background body of information. For our purposes, we can think of this body of information---the \emph{context set}---as representing what is mutually recognized as being taken for granted for the purposes of the conversation. The purpose of an assertion, on this picture, is to add information to the context set.\footnote{This picture is due, in its essentials, to Stalnaker. See  e.g. \citealt{stalnaker1973,stalnaker1978}. See also \citealt{lewis1979a}.}

For concreteness, we make a few working assumptions. First, we will think of a body of information as represented by a set of \emph{possible worlds}: the worlds that are not ruled out by that body of information. Second, we think of the \emph{content} of a declarative sentence as (determining) a set of possible worlds: those worlds in which the sentence is true. 

As the conversation evolves, so does the context set. After a successful utterance of a sentence $s$, the content of $s$ gets added to the context set---the context set is shrunk to exclude those in which $s$ is false.\footnote{And some more: the context set will, in general, get updated so that every world in it is one in which an utterance took place, and so on. This need not be the case: if participants are engaged in pretense, describing a situation in which their conversation is not taking place, information about the conversation will not get added to the context set. For our purposes, we can engage in the pretense that such conversations never take place.} 

Consider a simple example. Sue and Tom are having a conversation. The main item on the agenda: what Paolo has been up to in the past year. So far, they take each other to presuppose that Paolo lived in Cambridge last year. We can thus model the body of information they take for granted for the purposes of the conversation as the set of all worlds in which Paolo lived in Cambridge last year. Now, suppose Tom utters the sentence `Paolo did not pay taxes in Cambridge last year.' Assuming Tom's assertion is successful, the context set will now get updated with that piece of information. The updated context set will now contain all those worlds in which Paolo lived in Cambridge last year without paying taxes. 


\subsection{Putting things together}

Expressivists have seemed partial to the above picture of communication: not to the letter, mind you, but to the spirit.\footnote{[\textsc{insert references here}]} Their idea is quite simple. In a conversation about normative matters we are trying to influence what systems of norms others accept. Thus, when engaged in a conversation, there are now two parameters that we need to keep track of: the worlds compatible with what is mutually taken for granted for the purposes of a conversation, and the systems of norms compatible with what is mutually recognized as being accepted by participants in the conversation.\footnote{In the terminology \citealt{lewis1979}, expressivists think of the \emph{conversational score} of a conversation as determining (at least) a pair consisting of a set of worlds and a set of norms.} 

Go back to Sue and Tom. Their conversation has now moved on to issues of greater significance. They have moved on to discuss the moral status of a few types of actions. So far, they take each other to accept that drowning babies in Coca-Cola is wrong. We can thus model what is mutually recognized as accepted for the purposes of the conversation by a set of systems norms: those systems of norms that forbid drowning babies in Coca-Cola. The moral status of tax-evasion is not one that has been discussed yet, but Sue goes on to utter the sentence: `Tax-evasion is wrong'. If Tom does not object, the updated context set will now include only systems of norms that forbid tax-evasion. Every system of norms in that set will now forbid drowning babies in Coca-Cola and not paying taxes. 

When giving a full-theory of communication, expressivists will probably add some subtleties to this picture. They might want to model the context set as a set of \emph{pairs}---those world-norm pairs $\langle w, n \rangle$ such that Paolo lived in Cambridge in $w$ without paying taxes \emph{and} $n$ forbids drowning babies in Coca-Cola and not paying taxes. They can thus predict, as they should, that Tom and Sue now take for granted, for the purposes of the conversation, that something Paolo did was wrong.\footnote{[\textsc{insert reference to the wishful thinking objection here}].}

For our purposes, however, we can set those subtleties aside. For the questions we will raise for expressivism would arise even if we never engaged in conversation to exchange information about what the world is like.  

\subsection{The problem of communication}

An important task for any theory of communication is to make sense of the practice of conversation: to explain why communication takes place the way it does. For the case of descriptive discourse, the above account of communication can give such an explanation. In broad outline, the explanation could go like this:

\begin{quote}
There is a convention in place so that (\emph{i}) when I utter a sentence $s$, I thereby convey that I believe the content of $s$---what $s$ says; and (\emph{ii}) if you take me to be reliable on the subject matter at hand, you will take my believing that $s$ is true as a reason for you to believe that $s$ is true.\footnote{Cf. the `convention of truthfulness and trust' in \citealt{lewis1975}.} Since this is common knowledge, a successful assertion of $s$ will result in the ruling out of those worlds in which $s$ is false---it will now be mutually recognized that we are taking for granted, at least for the purposes of the conversation, that $s$ is true.
\end{quote}

Can the expressivist give a similar story for the case of \emph{normative} discourse? Can she make sense of the practice of assertion of normative claims? 

The expressivist cannot take the above explanation verbatim. After all, it relies on the notion of \emph{reliability}, and it is not obvious how an expressivist can make sense of this notion. Reliability is typically cashed out in terms of correspondence with how things are. You take me to be reliable if you take my believing that so and so as an indication that so and so is the case. You take Sue's beliefs about some subject matter to be a good guide to the truth, because you take the following schema to be generally true (with $p$ ranging over a suitably restricted domain): 

\ex.[] If Sue believes $p$, then $p$ is true.

It may be that expressivists can ultimately appeal to something much like reliability in making sense of our communicative practices. But pending such a story, the expressivist needs a different way of making sense of our practice of assertion. 

Although no alternative account has been given in full detail, expressivists have gestured toward a way of making sense of communication, one that generalizes to descriptive and normative discourse alike. The idea, introduced by Allan Gibbard, is to think of communication as primarily an exercise in \emph{coordination}. We are social creatures: we engage in conversation because coordinating on some particular family of attitudes---beliefs, say, or normative acceptances---is likely to help us meet our goals. 

To a first approximation, this seems like a good strategy for making sense of our communicative practices. The point of engaging in conversation about non-normative matters is to coordinate on a body of beliefs. The point of engaging in conversation about normative matters is to coordinate on a system of norms. We work out, as a community, what to think about the world. We work out, as a community, what system of norms to accept. Communication can be understood as a way to foster such coordination. 

Here is Allan Gibbard, explicitly making this point: 

\begin{small}
\begin{quote}
	The biological function of the mechanisms underlying our normative capacities is to coordinate. Hence the psychic mechanisms that produce normative judgments are not systems of natural representations, they are coordinating systems. Their biological function is not to put something in the head in correspondence with their subject matter; it is to coordinate what is in one person's head with what is in another's. \cite[p. 110]{gibbard1990}
\end{quote}
\end{small}
And Seth Yalcin, building on Gibbard's strategy to flesh out an account of communication friendly to his expressivist account of epistemic modal talk: 
\begin{small}
	\begin{quote}
[I]n modeling the communicative impact of an epistemic possibility claim, we construe the objective as one of coordination on a certain global property of one’s state of mind—the property of being compatible with a certain proposition—not one of coordination concerning the way the world is. \cite[p. 310]{yalcin2011}
	\end{quote}
\end{small}

Unfortunately, to account for \emph{our} communicative practices, coordination by itself will not do. The reason is that coordination is too cheap. If our goal in conversation is to coordinate on one of two systems of norms, flipping a coin should do just fine.  Mere coordination is what we are after when deciding what side on the road to drive on: there, a coin flip would have done the trick. There, what we want is that we all drive on the same side of the road, regardless of \emph{which} side that is. But we don't normally rely on coin flips when working out what to do, no more so than we rely on coin flips when working out what the world is like. 

This is not to say that coordination cannot play a part in a general account of communication, one that makes sense of both normative and non-normative discourse. But a full account must do more than point out the advantages of coordination, and the ways in which conversation can foster coordination. For there are much more efficient ways of achieving coordination, so a desire for coordination on our beliefs or our normative acceptances cannot explain the complexities of our communicative practices. In the next section, we look at the ways in which this idea of communication as coordination needs to be constrained if it is to yield an explanation of why communication works the way it does. 

% TODO 
%(fold) 
% Gibbard thinks coordination won't do: "This accounts for pressures towards consensus in ethical matters. For many sorts of normative judgments, though, such an account of disagreement isn’t to be had. Two egoists can agree that each ought rationally to act to frustrate the aims of the other, but this agreement is far from fostering coordination. A story rooted in coordination won’t work for belief either, since we don’t standardly need to coordinate our beliefs." Found here: http://philosophy.fas.nyu.edu/docs/IO/5882/GibbardPaper.pdf
%(end)



% On this picture, communication can be thought of as a matter of coordination on a body of information.

\section{Constraints on coordination}\label{coc}

In this section, we argue that expressivists are committed to the following claim: 

\ex.[(\textsc{c})] In any `normal' conversation, it is presupposed that there is a (unique) system of norms on which the participants' acceptances ought to converge.

A few remarks are in order. First, \emph{presupposition} here does not require unconditional acceptance. All the expressivist is committed to is that, while the conversation takes place, participants must take for granted \emph{for the purposes of the conversation} that there is a unique system of norms to which the conversation ought to converge. 

Second, (\textsc{c}) does not amount to a capitulation to a form of realism or objectivism. A realist will presumably endorse the following thesis: 

\ex.[(\textsc{c$^*$})] There is a unique system of norms (the `true' system of norms) to which, in any `normal' conversation, participants' acceptances ought to converge. 

But (\textsc{c$^*$}) is crucially distinct from (\textsc{c}), and neither one entails the other. It may be that there is a stronger version of (\textsc{c$^*$}) that realists are also committed to, e.g.

\ex.[] There is a unique system of norms such that, in any `normal' conversation, it is presupposed that the participants' acceptances ought to converge to that system of norms. 

And while this principle does entail (\textsc{c}), it is not entailed by it. 

Third, note that in endorsing (\textsc{c}) the expressivist need not impute ordinary speakers with any type of false belief. In presupposing that there is a unique system of norms to which our acceptances \emph{ought} to accept I am not, according to the expressivist, taking a stance on what the world is like. So the presupposition in (\textsc{c}) is compatible with a perfectly accurate view on what the world is like. 

Finally, a word on the qualifier `normal': by `normal' conversation (at \emph{t}) we mean one in which it is presupposed (at \emph{t}) that there is a point to communication. In other words, one in which it is presupposed that participants in the conversation intend to achieve coordination.

Our argument for (\textsc{c}) is somewhat complex. Before spelling it out in some detail, we will take a moment to briefly outline the main moves we will make along the way. 

\subsection{Overview of our main argument}

We use `$n$',`$s$', and `$c$' as variables ranging over systems of norms, participants in a conversation, and context sets, respectively. Also, we use `$\mathsf{Acc}_s$' to denote the set of systems of norms compatible with what $s$ accepts, and we use `$\ought$' to abbreviate the \emph{ought} operator. This allows us to introduce some abbreviations. Our conclusion, for example, is just the claim that the following is a presupposition of every `normal' conversation

\ex.[] \textsc{convergence}: $(\exists n)(\forall s) \ought (n \in \mathsf{Acc}_s)$


In arguing that \Last must be presupposed in any conversation,\footnote{Henceforth, we drop the qualifier `normal' for the sake of readability.} we rely on two main premises. The first is that the following is presupposed in every conversation with context set $c$:

\ex.[(\textsc{p1})] $\ought ((\exists c' \subset c)(\forall s)\mathsf{Acc}_s \subset c')$.

The second is that 

\ex.[(\textsc{p2})] $(\exists n)(\forall s)\ought(n \in \mathsf{Acc}_s)$

is also presupposed in any conversation. 

In plain English, the first premise is that, in every conversation, it is presupposed that the context set ought to be narrowed down. The second premise is that, in every conversation, it is presupposed that there is a system of norms that ought not be ruled out from any of the participant's acceptance state. 

Unsurprisingly, most of the action will take place in the argument that (\textsc{p2}) is a presupposition in any normal conversation. 

Once we establish that (\textsc{p1}) and (\textsc{p2}) must be presupposed in any conversation, we will have completed the hard work. For, as we will show, every context set that contains \textsc{p1} and \textsc{p2} is a context set that contains \textsc{convergence}. Thus, if (\textsc{p1}) and (\textsc{p2}) are presupposed in every conversation, our conclusion (\textsc{c}) must obtain. 

\subsection{The first premise}

\subsection{The second premise}

\subsection{Deriving the main conclusion}



\section{Morals}\label{dragon}

 

\printbibliography
\end{document}