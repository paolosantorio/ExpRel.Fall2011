\documentclass[11pt,article,oneside]{memoir}
\usepackage{apc-memoir}

\chapterstyle{article-5}
\pagestyle{apcgit}

\title{Communication for expressivists}
\author{\MakeLowercase{Alejandro P\'erez Carballo \amper Paolo Santorio}}
\myaffiliation{}
\myemail{}
\version{1.0}

\published{Partial working draft}

\begin{document}
\maketitle
	
\noindent

\section{Introduction}

\section{Expressivism and the problem of communication}

\subsection{Expressivism}

\subsection{Standard accounts of communication}

We will think of conversations as taking place against a background body of information. For our purposes, we can think of this body of information---the \emph{context set}---as representing what is mutually recognized as being taken for granted for the purposes of the conversation. The purpose of an assertion, on this picture, is to add information to the context set. 

For concreteness, we make a few working assumptions. First, we will think of a body of information as represented by a set of \emph{possible worlds}: the worlds that are not ruled out by that body of information. Second, we think of the content of a declarative sentence as (determining) a set of possible worlds: those worlds in which the sentence is true. 

On this picture, communication can be thought of as a matter of coordination on a body of information. 

\subsection{Putting things together}


\end{document}